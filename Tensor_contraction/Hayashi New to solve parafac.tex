\documentclass[10pt, draft]{article}
\usepackage{amsmath}
\usepackage[margin=1in]{geometry}
\usepackage[utf8]{inputenc}
\usepackage{amsfonts}
%\usepackage{physics}

\newcommand{\dd}[1]{\mathrm{d}#1}

\begin{document}

\author{karl}
\raggedright
\textbf{\Large{\begin{center}
New algorithm to solve parafac model\\
Hayashi and Hayashi \\
 \end{center}}}
 
 %%%%%%%%%%%%%%%%%%%%%%%%%%%%%%%%%%
 
 \section{introduction}
The explanation and demonstration of the usefulness of the parafac model is in the papers by Harshman 1970 , 1977, 1980 and kruskal 1981.\\
Kruskal gave computer program of parafac developed by harshman in 1980.  \\
fool proof algorithm to impolement computationally. 

\section{Fundamental formula}
 based on successive approximation.  effective for rapid comvergence of solution to apply a good first approximation.  \\
 calculate $T_{ii'}$ using multiplication of tensor elements for paiwise i, i' elements.  Assuming the columns of factor matrix C is orthogonal and B are normalized.
 \[T_{ii'} = \sum_s^S a_{is}a_{i's}\]
 similar solutions for Tjj'\\
 
 One can determine the a's and b's under the conditions that sum of $a_is^2/I=1$ and similar for b for s=1,2,...,S by calculating the vectors corresponding to the characteristic roots in descending order in the characteristic equation below.  (eigen value equation per pair of ii' and jj')\\
 
 After this process a's b's and c's are obtained by the least squares method presented the derivate of the least squares equation wrt C =0.\\
 c's are determined as the solution of a linear simultaneous equation.\\
 
 Thus for all k and r we have 
 \[ c_{kr} \sum_i \sum_j x_{i,j,k} a_{ir} b_{ir}\]
 
 where r twlls how many characteristic roots of a and b to take.  \\
 one can assume any new factor is comprised of the original factor plus the original factor times some Delta a(b or c)\\
 this means one can order any set of a b and c in terms of a0 b0 and c0 + ...\\
 one can order these original factors into a matrix A(r) which is proportional to the factors and find xhat = xijk - sum Aijk(r)\\
 the proportionality constant is for the adjustment of scale which results by using normalized a,b and c.  \\
 
 The corresponding xhat ~ the Aijk(r) * the delta terms, is considered and the least squares method is used to obtain the deltas.\\
 The derivative of the least squares equation by any factor is 0 (not in Aijk(r))\\
 Solve simultaneous linear equation\\
 process repreated using obtained a, b, c instead of a0, b0, c0.  continued until solution is determined.\\
 The convergence of $\Delta z \rightarrow 0$ shown by the succesive process of decreasing abs(delta z) so that the coming is based on the solutions of simultaneous linear equation.
 
 %%%%%%%%%%%%%%%%%%%%%%%%%%
 \section{determination of S}
 They pick a random s to loik at a nd determine goodness of fit wrt S.
 
 \section{Measure of goodness of fit}
 
 Have an error function from the 2-norm then do goodness is 1-error squared over variance squared.
 
 \section{discussion of pre-processing of data}
 
 Preprocessing can help with the PARAFAC analysis but can also hurt the solution.  There are rules for preprocessing in this paper.  
 
 \section{example}
 
\end{document}