%plan.tex%
% Plan of Study %

\section{Research Objectives}
The accurate prediction of certain chiroptical properties is crucial to assigning absolute stereochemical configurations to chiral compounds. While predictive methods of optical rotation are used quite often, a single rotational angle does not provide enough information to make conclusions about molecular structure. On the other hand, circular dichroism offers more information about the structural makeup of a molecule. Specifically, the simulation of electronic circular dichroism spectra is routinely employed by computational chemists and compared to experimental spectra to assist in the assignment of absolute configurations. However, a difficulty arises in that ECD only provides structural information near the excited chromophore in the molecule. 

    Vibrational circular dichroism\cite{Stephens1985} is another experimental spectroscopic technique widely used in the elucidation of chiral compounds in order to make stereochemical configuration predictions. VCD is similar to its ECD counterpart in that it measures the differential absorption of left- and right-circularly polarized light, but is a more promising alternative due to its ability to provide structural information about the entirety of the molecule. Therefore, it would be extremely useful to be able to simulate such spectra with highly accurate computational models. 

    VCD rotational strengths are obtained from the dot product of the transition electric-dipole moment and the transition magnetic-dipole moment, i.e.
        \begin{equation}
            R_{if} = Im\{\bra{i}\hat{\mu}\ket{f}\cdot\bra{f}\hat{m}\ket{i}\}
        \end{equation}
    where $\hat{\mu}$ is the electric dipole operator, $\hat{m}$ is the magnetic dipole operator, "Im" indicates that only the imaginary part of the expression is kept, and $R_{if}$ is the VCD rotational strength between vibrational states $i$ and $f$. While the electric dipole transition moments are straightforward to calculate, the magnetic dipole transition moments are much more difficult to compute. An advantageous approach is to calculate the specific transition moments from the derivatives of the electric and magnetic dipole moments, which are known as the atomic polar tensors (APT's) and atomic axial tensors (AAT's), respectively\cite{Stephens1985}.
    \begin{equation}
        \frac{\partial\mu_\beta}{\partial{R^\lambda_\alpha}} = \bigg[ \frac{\partial}{\partial{R^\lambda_\alpha}} \bra{\psi_0(\bm{r};\bm{R})}\hat{{\mu}_\beta}\ket{\psi_0(\bm{r};\bm{R})} \bigg]
    \end{equation}
    \begin{equation}
        \frac{\partial{m}_\beta}{\partial{\dot{R}^\lambda_\alpha}} = \bigg[ \frac{\partial}{\partial{\dot{R}^\lambda_\alpha}} \bra{\psi_0(\bm{r};\bm{\dot{R}})}\hat{{m}_\beta}\ket{\psi_0(\bm{r};\bm{\dot{R}})} \bigg]
    \end{equation}
    The previous equations are derived for the $\lambda$th nucleus, where $\alpha$ and $\beta$ represent Cartesian x-, y-, and z-components, and $\bm{R}$ and $\bm{\dot{R}}$ denote nuclear positions and velocities, respectively. In other words, these transition moments require the second derivative of the energy with respect to nuclear perturbations as well as electromagnetic perturbations. In order to increase computational efficiency and limit numerical precision errors, I plan to employ analytic second derivatives in order to calculate the APT's and AAT's.
     
    Currently, some popular quantum chemical programs (Gaussian, Dalton, ADF, PQS, CADPAC) possess the abilities to calculate VCD spectra, however, current implementations only allow for VCD calculations at the HF, MCSCF, and at most DFT levels of theory. While DFT calculations are often adequate to assign absolute stereochemical configurations, there are instances where high-accuracy methods are necessary. For example, the idea of "chirality transfer", or induced chirality, in which interactions between a chiral molecule and an achiral solvent cause the otherwise optically inactive vibrational modes of the solvent to be become active, cannot be accurately described by VCD simulations at the DFT level of theory, at least for pulegone in deuterated chloroform\cite{Debie2008,Nicu2009}. It is suspected that additional similar examples will continue to surface as theoretical VCD spectra calculations become more widespread. In such cases, it would be advantageous if high-level \textit{ab initio} method implementations of VCD calculations existed in a widely used computer code. 

    The main objective of my research will be to extend these predictive models first to MP2 and eventually to CC theories, ultimately allowing for CC simulations of VCD spectra. To the best of my knowledge, this will be the first instance of quantum chemical models for VCD at this level of theory. Since our group is a primary developer of CC methods in the quantum chemical package PSI4, all implementations will take place in PSI4. I suspect that derivations, implementations, and ample testing of the method will require a majority of my time as a graduate student. 
    
    One possible limitation of the method is the inherent lack of gauge invariance in standard CC theory. The later stages of the proposed research will focus on optimized orbital approaches as well as the inclusion of gauge invariant atomic orbitals in order to mitigate this issue. Additionally, the high-order scaling of CC theory with molecular size [e.g., $O(N^6)$ at the singles and doubles level] narrows the range of possible applications. However, eventual merging of the proposed methods with reduced-scaling techniques will permit large-scale VCD calculations that are currently intractable. By expanding the current capabilities of VCD computations, I hope to shed light on the fundamental connections between molecular geometry and chiroptical response, a bridge that is poorly understood in the chemistry community, eventually leading to the accurate prediction of optical activity in large, dynamically changing chiral molecules.
