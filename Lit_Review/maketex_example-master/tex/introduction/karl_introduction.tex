%introduction.tex%
% Introduction %

\section{Introduction}
Computational chemistry has developed as a tool to assist experimental and synthetic chemists.  The goal of theoretical and quantum chemistry is to develop mathematics to computationally predict the accurate properties of chemical systems using the statistical physics of electrons. Accuracy to an extremely small degree is important to correctly predict the small energy differences in electronic states or different geometries.\cite{schavitt 1977}  In order to model chemical systems physics casts the problem into a many-body integro-differential equation known as the many-body Schr{\"o}dinger equation.  These equations, in their raw form, are much too difficult to solve analytically.  The task of computational chemistry is to provide an approximate framework which preserves the physical nature and accuracy of the exact model, while reducing mathematical complexity.  There exists a number of methods applicable to different systems depending on the user's interest to balance computational effort and time, accuracy, and chemical system size.\\
The gold standard in computational chemistry are the coupled cluster (CC) methods which are typically used to calculate ground state calculations but require the most computational resources and time.  Hartree-Fock and its perturbation based extensions such as M{\/o}ller-Plesset (MP) methods can recover some electron correlation and can be extended to molecules with around 100 atoms.  Other methods such as density functional theory (DFT) and semi-empirical methods based on classical mechanics can solve even larger molecules though their accuracy can be poor.
