%introduction.tex%
% Introduction %

\section{Introduction}
    Enantiomeric pairs consist of a chiral molecule and its non-superimposable mirror image, which differ only in their configuration of atoms around a stereogenic center, axis, or plane. As such, enantiomers are characterized by their chirality, or three-dimensional handedness. The resulting pair of chiral molecules often exhibit distinct chemical behavior in chiral environments like the human metabolism, sometimes resulting in differing biological effects\cite{Eliel1994}. Consequently, pharmaceutical companies have a significant interest in the characterization of enantiomeric drugs, which make up over half of the approved drugs worldwide\cite{Nguyen2006}. These characterization techniques are also essential for organic chemists in their synthesis of chiral compounds in the laboratory, where absolute and relative configurations of stereogenic centers must be carefully controlled. Although procedures have been developed to consistently isolate enantiomerically pure samples, the assignment of absolute configurations poses a much larger task, especially when the molecules of interest exhibit numerous stereogenic centers. However, enantiomers also respond differently to left- and right-circularly polarized light in absorption, emission, refraction, and scattering\cite{Crawford2006,Crawford2007a,Crawford2012,Barron2004,Pecul2005}. These distinct responses may be used to assist in the assignment of absolute stereochemical configurations to enantiomerically pure samples, provided sufficient knowledge of their theoretical chiroptical spectra. Hence, theory plays a vital role in simplifying this characterization process. 

    Over the last several decades, \textit{ab initio} quantum chemical calculations have been employed to accurately and reliably predict various molecular properties, including chiroptical properties\cite{Crawford2012,Crawford2007a,Crawford2006,Pedersen1999}. Specifically, optical rotations and electronic circular dichroism (ECD) are regularly calculated and compared to experimental results in an attempt to assign absolute stereochemical configurations to optically active molecules. However, a difficulty arises in that these electronic spectroscopic techniques often offer limited knowledge about the complex molecular structure of chiral compounds. 

    A promising alternative to the aforementioned methods that provides additional structural information is vibrational circular dichroism (VCD)\cite{Stephens1985}, which measures the differential absorption of left- and right-circularly polarized light. Since its introduction, VCD has proven to be one of the most useful and reliable methods for assigning absolute configurations to chiral molecules\cite{Stephens2000}. Different approaches to calculate VCD rotational strengths have been thoroughly laid out in the literature by various authors\cite{Stephens1985,Nafie1997,Stephens2000,Bak1993,Bak1994,Amos1987}. The most efficient approaches have been successfully implemented into several popular computer codes at various \textit{ab initio} levels of theory, including Hartree-Fock (HF) theory\cite{Hartree1928,Fock1930}, multi-configurational self-consistent field (MCSCF) theory\cite{Roos1980}, and density functional theory (DFT)\cite{Hohenberg1964,Kohn1965}. 

    While VCD spectra simulations, in particular at the DFT level of theory, are responsible for numerous assignments of absolute configurations, shortcomings in the methodology undoubtedly exist. For instance, it has been reported that induced chirality between the molecule pulegone and deuterated chloroform solvent cannot be reliably investigated by DFT methods due to the prediction of incorrect signs of peaks in the VCD spectra\cite{Debie2008,Nicu2009}. While the use of a polarizable continuum model improves on the results, higher level \textit{ab initio} quantum mechanical methods are necessary to reconcile some of the discrepancies between computed and experimental spectra. However, no implementations of advanced many-body methods --- such as those based on perturbation theory\cite{Møller1934} or coupled cluster theory\cite{Coester1958,Coester1960,Cizek1966,Cizek1969,Bartlett1978,Pople1978} --- currently exist in any quantum chemical software package for computing VCD spectra.

    Therefore, I propose to develop high-level \textit{ab initio} methods to compute VCD rotational strengths. Ultimately, this necessarily requires the computation of the atomic polar tensors and atomic axial tensors, which involve the derivation of analytic second derivatives of the molecular energy with respect to external perturbations. I initially plan to derive and implement these equations for second-order M{\o}ller-Plesset perturbation theory (MP2)\cite{Møller1934} and eventually do the same for coupled cluster (CC) theory, allowing for CC simulations of VCD spectra. CC theory is among the most accurate quantum mechanical models and is often appropriately referred to as the "gold standard" of quantum chemistry\cite{Crawford2007}. I will implement these methods within the PSI4 open-source quantum chemical package\cite{Turney2012}, to which the Crawford group is a primary contributor. Close comparisons with empirical data, made possible through long-standing collaborations with experimentalists, will allow for robust testing of the approach. 
