%outline.tex

\begin{outline}[enumerate]
    \1 Introduction
        \2 Big picture of analytic derivatives
            \3 Usefulness in quantum chemical calculations
                \4 Properties of interest to all chemists
            \3 Advantages of analytic versus. numerical differentiation
                \4 Numerical accuracy (obviously)
                \4 Increased computational efficiency
            \3 Disadvantages
                \4 Computational cost with higher order derivatives
    \1 \sout{Electronic Structure Theory}
        \2 \sout{Schr\"{o}dinger Equation}
        \2 \sout{Hartree-Fock Theory}
            \3 \sout{Single Determinant Representation minimized with constraint that spin orbitals remain orthonormal}
            \3 \sout{Shortcomings - Mean field approximation (no electron correlation energy)}
        \2 \sout{M{\o}ller-Plesset Perturbation Theory}
            \3 \sout{Treat electron correlation as perturbation to the HF SCF solution}
        \2 \sout{Configuration Interaction}
            \3 \sout{Linear combination of determinants}
            \3 \sout{Variational, but not size consistent}
        \2 \sout{Coupled Cluster Theory}
            \3 \sout{Exponential form of the wave function}
            \3 \sout{Size Consistent, but no variational}
    \1 Optical Response
        \2 Response Theory
            \3 Exact Response
            \3 Linear Response
            \3 Quadratic and Higher Order Response
        \2 Properties
    \1 Analytical Gradient Theory
        \2 Wigner's $2N + 1$ Rule
        \2 First Derivatives
            \3 Derivations
            \3 Associated Properties
        \2 Second Derivatives
            \3 Derivations
            \3 Associated Properties
    \1 Proposed Plan of Study
        \2 Implementation
        \2 Calculation of properties
            \3 CC Simulations of Vibrational Circular Dichroism (VCD) Spectra
            \3 CC Simulations of Magnetic Circular Dichroism (MCD) Spectra
            \3 Non-linear Optical Properties (hyperpolarizabilities)
\end{outline}

