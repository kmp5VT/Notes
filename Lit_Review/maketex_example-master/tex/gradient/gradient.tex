%gradient.tex%
% Analytic Gradient Theory %

\section{Analytical Gradient Theory}
    There is a deep underlying connection between response theory and gradient theory. In fact, the properties that are calculated with response functions can also be calculated with energy derivatives. Both approaches deal with a system that is subject to some perturbation, however, the difference between the two lies in the nature of the perturbation. Perturbations are often thought of as external electric or magnetic fields, but changes in nuclear coordinates or any other slight change to the system may also be considered to be perturbations. Due to the Fourier transformed perturbation in equation (\ref{eq:FT}), response functions can be used to calculate dynamic properties that depend on the frequency $\omega$ of an external field. Moreover, response functions can also be used to calculate properties generated by a static field by setting $\omega$ equal to zero. On the other hand, derivatives of the energy are used to calculate static field properties because their formulation inherently lacks a frequency dependence. For example, the dynamic polarizability of a molecule due to a frequency-dependent external electric field may be calculated using the linear response function, whereas an induced electric dipole moment (static polarizability) can be calculated using the second derivative of the energy with respect to the external electric field.

    Ultimately, energy derivatives can be calculated numerically or analytically, but analytical differentiation has a few major advantages when compared to numerical differentiation. The first advantage is obviously numerical accuracy. The accuracy of numerical differentiation is highly dependent on the increment size, resulting in errors if the increments are too small or too large, whereas, analytical derivatives avoid approximations and are only limited by numerical precision of the computer. The second advantage to using analytical derivatives is computational efficiency. For example, all analytical first derivatives of the energy with respect to nuclear positions can be calculated with a comparable computational cost to calculating the energy in the first place. An additional advantage to analytic gradients vs. numerical is that you can't use the latter for frequency-dependent properties. The main disadvantage of analytical derivatives stems from the difficulties in implementation, especially with respect to storage requirements and parallelization\cite{Pulay2014}.
\subsection{History}
    The first derivations of HF first and second derivatives was presented by Brato{\v{z}}\cite{Bratoz1958} in 1958, however, it remained unpopular due to its early introduction. 10 years later, Gerratt and Mills\cite{Gerratt1968} proposed the idea of calculated force constants as analytical derivatives of Hellmann-Feynman forces, which ultimately proved to be an unreliable approach. Pulay reintroduced analytical first derivatives in 1969\cite{Pulay1969} suggesting that analytical derivatives should stop at the first derivative, and force constants should be calculated by numerical differentiation of the first derivatives. Pople et al.\cite{Pople1979} were the first to report a practical implementation of second derivatives for HF and MP2, representing the first extension of analytical derivatives to a dynamical electron correlation method. Since, analytical first and second derivatives have been extended to CI\cite{Krishnan1980,Brooks1980}, DFT, and CC at various levels of theory\cite{Scheiner1987,Lee1991,Koch1990a,Gauss1997} made practical through the Z-vector technique\cite{Hoffmann1984}.
\subsection{Wigner's $2N+1$ Rule}
    Regardless of whether response theory or analytical gradient theory is used, certain knowledge about the wave function is necessary in order to calculate certain properties. This is a consequence of the fact that perturbation theory and analytical gradient theory are both subject to Wigner's $2N+1$ rule\cite{Wigner1935}, which states that knowledge of the wave function through $N$th order provides enough information to evaluate the ($2N+1$)th perturbation contributions, or energy derivatives. This means that the zeroth order derivative of the wave function is adequate to determine the first derivative of the energy. A consequence of this is that the computational cost of calculating all nuclear first derivatives is similar to the cost of the energy. Likewise, the first order derivative of the wave function will suffice to calculate the second and third derivatives of the energy. In CC theory, Wigner's $2N+1$ rule applies to the T amplitudes in that knowledge of the T amplitudes through $N$th order is sufficient for the ($2N+1$)th perturbative corrections to the energy. There is an analogous $2N+2$ rule for the $\Lambda$ amplitudes as well\cite{Eriksen2014}. 
\subsection{Associated Properties}
    The calculation of analytical energy derivatives has revolutionized quantum chemistry, making electronic structure calculations accessible and useful for a sizable number of chemists. Table 1 lists some observable quantities and their associated energy derivatives\cite{Pulay1995}. 
    \begin{table}
    \caption {Energy Derivatives and Observables}
    \small
    \begin{center}
        \begin{tabular} {| c | p{13cm} | }
            \hline & \\
            \multicolumn{1}{|c|}{\bfseries Quantity} & \multicolumn{1}{|c|}{\bfseries Observable} \\ & \\
            \hline & \\
            $\frac{\partial E}{\partial R}$ & forces on the nuclei; critical points on the potential energy surface (minima and saddle points) \\ & \\
            $\frac{\partial^2 E}{\partial R_i \partial R_j}$ & force constant, fundamental vibrational frequencies; infrared and Raman spectra; vibrational amplitudes; vibration-rotation couplings \\ & \\
            $\frac{\partial^3 E}{\partial R_i \partial R_j \partial R_k}$ & cubic force constants; anharmonic contributions to vibrational frequencies; anharmonic contributions to vibrational averages \\ & \\
            $\frac{\partial^4 E}{\partial R_i \partial R_j \partial R_k \partial R_l}$ & quartic force constants; anharmonic contributions to vibrational frequencies \\ & \\
            $\frac{\partial E}{\partial F}$ & dipole moment; intermolecular forces \\ & \\
            $\frac{\partial^2 E}{\partial F_\alpha \partial F_\beta}$ & polarizability; intermolecular forces; light scattering \\ & \\
            $\frac{\partial^3 E}{\partial F_\alpha \partial F_\beta \partial F_\gamma}$ & (first) hyperpolarizability; second harmonic generation \\ & \\
            $\frac{\partial^2 E}{\partial R_i \partial F_\alpha}$ & dipole moment derivative; infrared intensities \\ & \\
            $\frac{\partial^3 E}{\partial R_i \partial F_\alpha \partial F_\beta}$ & polarizability derivative; Raman intensity \\ & \\
            $\frac{\partial^3 E}{\partial R_i \partial R_j \partial F_\alpha}$ & electrical anharmonicity, intensities of overtones in the infrared spectrum; vibrationally averaged dipole moments; electric field effect on the harmonic force constants \\ & \\
            $\frac{\partial E}{\partial B}$ & magnetic dipole moment \\ & \\
            $\frac{\partial^2 E}{\partial B_\alpha \partial B_\beta}$ & magnetic susceptibility \\ & \\
            $\frac{\partial^2 E}{\partial B_\alpha \partial \mu_\beta}$ & NMR chemical shielding \\ & \\
            $\frac{\partial^3 E}{\partial R_i \partial F_\alpha \partial B_\alpha}$ & infrared optical rotatory power \\ & \\
            \hline
        \end{tabular} \\
        \bigskip
        \textit{Notation: R: nuclear coordinate; F: electric field; B: magnetic flux density; $\mu_{\beta}$: the $\beta$ component of the magnetic moment.}
    \end{center}
    \end{table}
    While many chemists are only interested in black box calculations that make use of only the simplest energy derivatives (e.g. forces and force constants for geometry optimizations), a wide range of chemical properties can be predicted with significant accuracy using energy derivatives. For example, energy derivatives are essential to calculating magnetic properties such as magnetic dipole moments, NMR spectra, magnetizability, etc.\cite{Pulay2014} 

