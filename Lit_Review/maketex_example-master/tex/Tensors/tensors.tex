\section{Tensor Algebra Methods to Reduce Computational Complexity in Quantum Chemistry}
	A tensor is a multidimensional array; an N-th order tensor is an element of the tensor product of N vector spaces.
	A first order tensor is an array, a second order tensor is a matrix and tensor of order three or higher is referred to as a higher-order tensor.\cite{Kolda2008}. Tensors are naturally applied to single reference quantum mechanics: operators such as \textbf{F} can be expressed in terms of two electron coordinate products and form second order tensors while other operators and amplitudes such as the coulomb repulsion operator, $\hat{J}$, and the cluster operator amplitudes, $t^{ij\dots}_{ab\dots}$, can be expressed as higher-order tensor. Using tensor's as storage devices creates a problem, known as the "curse of dimensionality", where storage and computational processing required to solve a problem scale exponentially with dimension.
	%This extension therefore means as number of electrons increase, there is an exponential increase in the amount of storage and computational processing required to solve a given problem. This problem is referred to as the "curse of dimensionality" %find a curse of dimensionality paper!%
	To overcome this curse requires one to discover the underlying structure in ones data allowing one to reduce storage requirements and to redesign algorithms to scale with the structure of the data.

	The goal of any tensor decomposition is to reduce the complexity of a tensor via the underlying form of ones data. The result of a tensor decomposition provide information on the relative importance and weighting of individual vector spaces. Direct methods to compute second order, matrix decompositions such as the singular value (SVD), lower-upper (LU), and Jordan decomposition, have been around for quite some time. Though, interests to decompose higher order tensors didn't develop until 1927 with Hitchcock's idea of a tensor to be a polyadic sum of products\cite{Hitchcock1928,Hitchcock1927} and later Cattells's idea of a multi-way model in 1944\cite{Cattell1944,Cattell1952}. These ideas would later be used to develop canonical product(CP) (CANDECOMP/PARAFAC canonical decomposition / parallel factor decomposition) \cite{Carroll1970,Harshman1970} and Tucker decompositions\cite{Tucker1966}. In order to apply ab initio quantum mechanics to larger systems and circumvent the "curse of dimensionality" it is necessary to take advantage of matrix and higher order tensor decomposition approximations and to redesign canonical algorithms using tensors in decomposed form. To follow are theoretical chemist's current tools to approximate and reduce the complexity of large systems while preserving accuracy.
	\subsection{Cholesky Decomposition}
		The Cholesky decomposition (CD) was first applied to quantum chemistry and specifically the two electron integral (TEI) tensor in 1977 by Beeble and Linderberg\cite{Beebe1977} when the authors realized that, coupled with the positive definite nature of the integrals values, one could to reorder the higher-order tensor into a lower order object and perform a matrix decomposition. What makes the CD special is that it can remove small and zero eigenvalues without calculating the entire matrix, providing computational savings. CD has been in conjunction with two electron geminal implementation, derivative integrals and more recently has been applied to large scale TEI decomposition\cite{Aquilante2011}.

		CD works by using a partial (LU) decomposition of any two electron tensor recast into a symmetric positive definite matrix 
			\begin{equation}
				\begin{aligned}
					\label{eqn:CD}
					M_{\mu\nu, \gamma\sigma} = \int \int \rho_{\mu\nu}(r_1)\hat{M}(r_1,r_2) \rho_{\gamma\sigma} dr_1 dr_2 \equiv (\rho_{\mu\nu}|\rho_{\gamma\sigma})
				\end{aligned}
			\end{equation}
		where $\rho_{\mu\nu} = \phi_\mu\phi_\nu$ is an orbital density product and $\hat{M}(r_1,r_2)$ is some two electron operator. The goal of the CD is to express M as 
		%With the goal of expressing
			\begin{equation}
				M = BB^T
			\end{equation}
		This expression can be approximated to some extent, $\gamma$, with elements of M expressed as%therefore elements of M can be expressed as
			\begin{equation}
				\begin{aligned}
					M_{\mu\nu, \gamma\sigma} \approx \sum_{p=1}^{P} B_{\mu\nu}^P B_{\gamma\sigma}^P &= \sum_{p=1}^{P} (\rho_{\mu\nu}|B_p)(B_p|\rho_{\gamma\sigma})\\
					&= \sum_{pq} (\rho_{\mu\nu}|b_p)(\hat{M}(r_1,r_2)^{-1})_{pq}(b_q|\rho_{\gamma\sigma})
				\end{aligned}
			\end{equation}
		where P is the rank of the decomposition which depends on $\gamma$. A comprehensive CD algorithm is presented by Epifanovsky et al\cite{Epifanovsky2013} which can be used to find optimal Cholesky basis, $b_p$, for a given $\hat{M}(r_1,r_2)$
	\subsection{Density Fitting}
		Density fitting (DF) is an specific application of CD where a canonical optimized Cholesky basis is used to decompose the TEI tensor into two order three tensors. The roots of DF have been grounded in Coulomb\cite{Ten-no1995,Vahtras1993} and Exchange\cite{Weigend2002} fitting in Hartree-Fock and has been applied to MP2\cite{Feyereisen1993}, CCSD(T)\cite{Rendell1994} and even explicitly correlated methods\cite{Manby2003}. The derivation of DF to proceed will be based on equations presented by Werner et al.\cite{Werner2003}. The goal of DF is to decompose the TEI tensor
			\begin{equation}
				\begin{aligned}
			\olap{\mu\gamma}{\nu\sigma} = (\mu\nu|\gamma\sigma) &= \int\frac{\phi_\mu(r_1) \phi_\nu(r_1) \phi_\gamma(r_2)\phi_\sigma(r_2)}{r_{12}} dr_1 dr_2\\
			&= \int \frac{\rho_{\mu\nu}(r_1)\rho_{\gamma\sigma}(r_2)}{r_{12}}dr_1 dr_2
				\end{aligned}
			\end{equation}
		one electron densities, $\rho_{\mu\nu}(r) = \phi_\mu(r) \phi_\nu(r)$, can then be approximated as 
			\begin{equation}
				\bar{\rho}_{\mu\nu}(r) = \sum_A^{N_{\text{fit}}} d^{\mu\nu}_A \chi_A(r)
			\end{equation}
		where $\chi_A(r)$ are fitting basis functions and expansion coefficients, $d^{\mu\nu}_A$, are expressed as %can be found by minimizing the functional 
			%\begin{equation}
			%	\Delta_{\mu\nu} = \int dr_1 \int dr_2 \frac{(\rho_{\mu\nu}(r_1)-\bar{\rho}_{\mu\nu}(r_1)) - (\rho_{\mu\nu}(r_2) - \bar{\rho}_{\mu\nu}(r_2))}{r_{12}}
			%\end{equation}
		%which provides 
			\begin{equation}
				d^{\mu\nu}_B = \sum_B (\mu\nu|A)[J^{-1}]_{AB}
			\end{equation}
		where 
			\begin{equation}
				(\mu\nu|A) = \int dr_1 \int dr_2 \frac{\phi_\mu(r_1)\phi_\nu(r_1)\chi_A(r_2)}{r_{12}}
			\end{equation}
		The term J is chosen to be some metric, here it is defined as the coulomb metric\cite{Dunlap1977,Dunlap1979}
			\begin{equation}
				J_{AB} = \int dr_1 \int dr_2 \frac{\chi_A(r_1) \chi_B(r_2)}{r_{12}}
			\end{equation}
		other metrics have been proposed\cite{Baerends1973} and though they are less accurate, these metrics are computed more quickly than the Coulomb metric.\\
		This allows one to express the TEI as 
			\begin{equation}
				(\mu\nu|\gamma\sigma) = \sum_B d^{\mu\nu}_B (B|\gamma\sigma) = \sum_{AB} (\mu\nu|A)[J^{-1}]_{AB}(B|\gamma\sigma)
			\end{equation}
		transforming $J^{-1} = J^{-1/2}J^{-1/2}$ allows one to store the TEI as two order three tensors, reducing storage requirements from $\mathcal{O}(N^4)$ to $\mathcal{O}(N^2\cdot N_{fit}) \approx \mathcal{O}(N^3)$; $N_{fit}$ typically scales linearly with basis set. Using DF Reduced scaling algorithms to calculate values such as the HF Coulomb term, $\hat{J}$, and AO to MO integral transformations et al have been developed.  %Additionally one finds reduction in computational effort in calculations such as the Coulomb term, $\hat{J}$, in HF and transforming integrals from the AOs to MOs in MP2 calculations.\\
		
		To further reduce %the complexity and storage 
		scaling of DF one can choose to use a subset of the full auxiliary basis. Original construction of subsets was developed using distance based domains. Unfortunately, this led to discontinuities on the potential energy surface. Recently it has been shown that a better auxiliary subset includes, for a given density, $\rho_{\mu_a\nu_b}$, fitting functions on either center $a$ or $b$, $|A_{(ab)})$; %. This method is 
		referred to as concentric atomic density fitting (CADF). This idea combined with localization methods and inclusion of exact semi-diagonal terms has been shown to reduce complexity in the calculation and storage of the coulomb, $\hat{J}$, and exchange term, $\hat{K}$, in HF by Hollman et al\cite{Hollman2014}
	\subsection{Direct Tensor Decomposition methods}
		%In the case of matrices, applications of decomposition methods are straightforward and imply the transformation to some canonical form based on the rank of the matrix. 
		Matrix decomposition methods are straightforward and refer to the transformation to some canonical matrix product representation. 
		The extensions of decompositions to higher order tensors is not simple. The rank of a tensor is defined as the smallest number of rank one tensors that generate the tensor as its sum, where a rank one tensor is defined as
			\begin{equation}
				\mathit{X} = a^{(1)} \otimes a^{(2)} \otimes \dots \otimes a^{(N)}
			\end{equation}
		where
			\begin{equation}
				\begin{aligned}
					\mathit{X} &\in \mathbb{R}^{I_1I_2\dots I_N} \\
					a^{(1)} \in \mathbb{R}^{I_1}, \quad a^{(2)} &\in \mathbb{R}^{I_2}, \quad \dots, \quad a^{(N)} \in \mathbb{R}^{(N)}
				\end{aligned}
			\end{equation}
		and a rank R tensor, $\mathit{U}$, can be defined in either the canonical format (CP)
			\begin{equation}
				\mathit{U} = \sum_{r=1}^R \lambda_r a^{(1)}_r \otimes a^{(2)}_r \otimes \dots \otimes a^{(N)}_r \quad \lambda_r \in \mathbb{R}
			\end{equation}
		where $\lambda$ is the normalization of the normalized vectors $a^{(l)}_i$ and one can then define a factor matrix as 
			\begin{equation}
				A^{(l)} = [a^{(l)}_1,\dots a^{(l)}_r]\quad l\in\{1,2,\dots,N\}
			\end{equation}
		such that 
			\begin{equation}
				A \in \mathbb{R}^{I_l R}
			\end{equation}
	 	%It is not necessary to require each factor matrix have the same rank; thus one can define the Tucker decomposition format (also refered to as the higher order singular value decomposition, HOSVD)
	 	Or $\mathit{U}$ can be defined in the Tucker format (also referred to as the higher order singular value decomposition, HOSVD), where each factor matrix is not required to have the same rank, R,
			\begin{equation}
				\mathit{U} = \sum_{\alpha_1}^{r_1} \dots \sum_{\alpha_N}^{r_N} \beta_{\alpha_1\dots\alpha_N}a_{\alpha_1}^{(1)}\otimes \dots \otimes a_{\alpha_N}^{(N)}
			\end{equation}
		where $\{ a_{\alpha_i}^{(l)}\}$ is a set of $R_l$ orthonormal vectors and $\beta \in \mathbb{R}^{r_1,r_2,\dots r_n}$ is the Tucker core tensor. Figure's 2 and 3 depict diagrammatically the CP and Tucker format
	  \begin{figure}
		\centering
			\begin{minipage}{.5\textwidth}
			  \centering
			  \includegraphics[width=1\linewidth]{./pics/CP_picture.jpg}
			  \captionof{figure}{Representation of CP format}
			  \label{fig:Figure 2}
			\end{minipage}%
			\begin{minipage}{.5\textwidth}
			  \centering
			  \includegraphics[width=1 \linewidth]{./pics/Tucker_picture.jpg}
			  \captionof{figure}{Representation of Tucker format}
			  \label{fig:Figure 3}
			\end{minipage}
		\end{figure}

	  Unlike matrix decompositions, there are no concise method to calculate the rank of a tensor, solving the rank is an NP hard problem\cite{Hastad1990}. Though there are many schemes which can solve for the approximate rank of a tensor, $\mathit{T}$, by iteratively minimizing a series of non-linear equations\cite{Kolda2008}
	  	\begin{equation}
	  		\begin{aligned}
	  			\|\mathit{T} - \mathit{U}\| < \epsilon
	  		\end{aligned}
	  	\end{equation}
	  where $\mathcal{U}$ is defined using canonical or Tucker format. 

	  Historically, the Tucker decomposition is linked to complete active-space self-consistent field (CASSCF) method\cite{Roos1980}, where the decomposition of excitation amplitudes yields optimized orbitals and the CP decomposition can be linked to full CI\cite{Bell2010} (FCI) where methods such as perfect pairing approach can be considered rank one tensor approximations to the FCI tensor. Applications of the CP decomposition to FCI recently resurfaced\cite{Uemura2012,Bohm2016}. %. The idea of using CP decomposition for FCI has recently resurfaced\cite{Uemura2012,Bohm2016}. 
	  Today, there is an effort to make use of the tensor element sparsity that naturally occurs as dimension increases to decompose tensors in canonical ab initio methods. In work presented by Benedikt et al\cite{Benedikt2011,Benedikt2013,Benedikt2013a,Benedikt2014}, post-HF operator and amplitude tensors are decomposed to compute MP2 and CCD using CP format for example
	  	\begin{equation}
	  		(\mu\nu|\rho\sigma) = \sum_r^R \chi^{(\mu)}_r \otimes \chi^{(\nu)}_r \otimes \chi^{(\rho)}_r \otimes \chi^{(\sigma)}_r
	  	\end{equation}
	  Using this form the authors developed equations to preserve decomposed form and rank. These methods allow for reduced complexity in storage with out significant trade-off in accuracy. In this form efforts to compute a single index contraction between two tensors of order $d$ and $f$ with dimension of each order $N$ decomposed to rank $R_1$ and $R_2$ are reduced from scaling as $\mathcal{O}(N^{d + f -1})$ to $\mathcal{O}(N \cdot R_1 * R_2)$. 
	  %and reduce the computational effort to perform any d-th order tensor contraction from $\mathcal{O}(d)$ to $K \cdot d \cdot R1 \cdot R2$ where K is the number of orbitals, d is the dimension of the tensors, and R1 and R2 are the ranks of each tensor. 
	  Unfortunately, finding the optimal rank and CP decomposition of tensors, such as the TEI, is non-trivial and costly and tensor contractions increases storage requirements, though it is possible to perform decompositions to reduce the contracted tensor rank. Therefore, implementation of CP decomposed post-HF methods are not yet desirable.  Efforts to implement a fast tensor compression algorithm to reduce the effort of computing the CP decomposition will be discussed in the research sections.

	  In work presented by Bell et al\cite{Bell2010} truncated HOSVD is employed to decompose the MP2 method $T_2$ amplitude expression, 
	  	\begin{equation}
	  		T_2(i,a,j,b) = \frac{(ia|jb)}{\epsilon_i + \epsilon_j - \epsilon_a - \epsilon_b}
	  	\end{equation}
	  This author showed that HOSVD could reduce storage of $T_2$ amplitudes for MP2 energy recovery from 85 to 99\%. Additionally, they showed that orbital active spaces obtained through the HOSVD coincide with physical intuition based on how the tensor is unfolded, in the first step of the HOSVD algorithm. Though the HOSVD does have some downsides, first HOSVD alone does not provide an optimal basis in terms of energy recovery it must therefore be coupled with other tensor decompositions and the algorithm to compute the HOSVD scales asymptotically as $\mathcal{O}(N^5)$\cite{Bell2010}.
	\subsection{Tensor Hypercontraction}
		Tensor Hypercontraction (THC) was introduced in 2012 by Hohenstein, Parrish and Martinez\cite{Schutski2017}. THC can be though of as a tensors decompositions applied to a DF decomposition, though in practice a CD or DF is not required. The THC for the TEI for example can be formulated a number of ways. As an example a TCH on a general four index tensor (FIT) will be derived. THC's goal is to recompose the FIT as a connected product of matrices 
			\begin{equation}
				V_{pqrs} \approx W_{p,\alpha} W_{q,\alpha} X_{\alpha\beta} W_{\beta, r} W_{\beta,s}
			\end{equation}
		First the step is to rewrite the FIT as a two index tensor
			\begin{equation}
				V_{pqrs} = V_{pq,rs}
			\end{equation}
		Then using an SVD one can express the two index tensor as
			\begin{equation}
				V_{pq,rs} = U_{pq,\lambda}S_{\lambda,\lambda} V^T_{\lambda, rs}
			\end{equation}
		where $\lambda$ is the rank of the decomposition, $U$ and $V^T$ are unitary matrices and $S$ is the singular value matrix. Here one may choose to use a truncated SVD or CD or DF. The singular values are then rerepresented as $S = S^{1/2} S^{1/2}$ and multiplied into the left and right singular vectors.
			\begin{equation}
				V_{pq,rs} = \tilde{U}_{pq, \lambda} \tilde{V}^T_{\lambda,rs}
			\end{equation}
		If one uses the CD or DF, a matrix roots of the overlap, $J^{1/2}$ of $J_{\lambda,\lambda}$ must be found to using an eigenvalue decomposition or SVD. Next, a CP decomposition is performed on the three index tensors $\tilde{U}$ and $\tilde{V}$
			\begin{equation}
				\begin{aligned}
					\tilde{U}_{pq, \lambda} = W_{p,\alpha} W_{q, \alpha} W_{\alpha, \lambda}\\
					\tilde{V}^T_{\lambda,rs} = W_{\lambda, \beta} W_{\beta, r} W_{\beta, s}
				\end{aligned}
			\end{equation}
		where $\alpha$ and $\beta$ are the rank of the CP decomposition. So far only applications where $\alpha = \beta$ have been studied. Finally the terms $W_{\alpha, \lambda} \text{ and } W_{\lambda, \beta}$ are contracted and one finds
			\begin{equation}
				V_{pqrs} = W_{p, \alpha} W_{q, \alpha} X_{\alpha, \beta} W_{\beta, r} W_{\beta, s}
			\end{equation}
		where
			\begin{equation}
				X_{\alpha, \beta} = W_{\alpha, \lambda} W{\lambda, \beta}
			\end{equation}
		THC has been used in the field to represent the electron interaction potentials in CC2 methods and to decompose the TEI used to calculate CCSD and FCI energies. In work presented by Hummel et al\cite{Hummel2016} using THC the scaling of distinguishable CCD or linearlized CCSD from $\mathcal{O}(N^6)$ to $\mathcal{O}(N^5)$ and in work presented by Schutski et al\cite{Schutski2017}using THC scaling of CCSD was reduced to $\mathcal{O}(N^4)$. Schutski also presents a direct THC method which allows TEI decomposition to scale as $\mathcal{O}(N^5)$ using the SVD or $\mathcal{O}(N^4)$ using a DF scheme while preserving accuracy of \textasciitilde.5 millihartree .
	\subsection{Orbital localization methods}
		A non-obvious method to reduce the complexity of tensors is to define new more compact occupied and unoccupied orbital sets, such is the basis for the projected atomic orbitals (PAO), pair natural orbitals(PNO) and orbital specific virtual (OSV) methods. Conveniently, unitary transformations of the molecular orbital space which do not mix occupied and unoccupied orbitals commute with all observable operators\cite{SzaboAttila1982} and these transformations can be used in orbital localization correlation (LC) methods. There are many developed orbital localization schemes such as Boys and Pipek-Mezey\cite{Boughton1993} among others which are utilized by PNO, PAO and OSV methods. In all the following methods MO are optimized using HF, though other optimizations are possible. Localized occupied MO's (OMO) will be denoted i,j,k and canonical unoccupied MOs (UMO) will be denoted a,b,c, non-canonical UMO's will be denoted r,s,t. All the following methods start by localizing the set of canonical OMO's. Below is a formula to generate a new occupied electron pair specific UMO's 
			\begin{equation}
				\ket{r^{ij}} = \sum_a \ket{a}R^{ij}_{ar}
			\end{equation}
		where $R^{ij}_{ar}$ is a pair specific transformation matrix. Using this occupied electron pair specific UMO one can transform the $T_1$ and $T_2$ amplitudes%This allows one to transform $T_1$ and $T_2$ amplitudes as 
			\begin{equation}
				t^i_a = \sum_{r\in[ii]} R^{ij}_{ar}t^i_r
			\end{equation}
			\begin{equation}
			t^{ij}_{ab} = \sum_{r\in[ij]} R^{ij}_{ar}t^{ij}_{rs}R^{ij}_{bs}
			\end{equation}
		In this format, correlation amplitudes and residual equations can be redefined and if possible reduced using domain approximations based on occupied electron pair distances\cite{Yang2012}.
		The PAO method works by projecting AO basis functions against the UMO's\cite{Pulay1983}
			%\begin{equation}\label{proj_AO}
			%	\ket{\tilde{\phi}_\mu} = \left(1- \sum_{i=1}^{m} \ket{(\chi_i)_L}\bra{(\chi_i)_L} \right) \ket{\phi_\mu} = \sum_{\rho=1}^N \ket{\phi_\rho}\tilde{R}_{\rho\mu}
			%\end{equation}
			%where the expansion coefficient of the projected functions $\tilde{\phi}_\mu$ in AO basis $\{\phi_\mu\}$ is given by 
			%	\begin{equation}
			%		\mathbf{\tilde{R}} = 1 - \mathbf{D} \cdot \mathbf{S}
			%	\end{equation}
			%where \textbf{D} and \textbf{S} are obtained during HF procedure\cite{Hampel1996}
			\begin{equation}
				\ket{r} = \sum_a \ket{a}R_{ar}
			\end{equation}
		where
				\begin{equation}
					R_{ar} = \olap{a}{\phi_r}
				\end{equation}
		This type of localization ensures the unoccupied space be orthogonal to the occupied space, but vectors in the unoccupied space are not orthogonal. PAO implementation has large impact in its implementation in CCSD(T), equation of motion CCSD, and more recently R12 methods by Werner et al\cite{Riplinger2013}. The number of PAOs to obtain accurate recovery of correlation energy (>99\%) grows linearly with size of the basis set per atom and domain sizes are asymptotically independent of molecule size.

		In PNO methods $R^{ij}_{ar}$ is defined by diagonalizing the MP2-like density matrix\cite{Yang2012,Neese2009}
			\begin{equation}
				D^{ij} = \frac{1}{1+\delta_{ij}}(\tilde{T}^{ij}T^{ij} + \tilde{T}^{ij}T^{ij^\dagger})
			\end{equation}
		where
			\begin{equation}
				T^{ij}_{ab} = \frac{\olap{ij}{ab}}{\epsilon_i + \epsilon_j - \epsilon_a - \epsilon_b}
			\end{equation}
			\begin{equation}
				\tilde{T}^{ij}_{ab} = 2T^{ij}_{ab} - T^{ji}_{ab}
			\end{equation}
		such that
			\begin{equation}
				D^{ij}R^{ij}_r = n^{ij}_rR^{ij}_r
			\end{equation}
		where $n^{ij}$ is the natural occupation number. Thus PNOs can be expanded in the basis of UMO or vice versa as
			\begin{equation}
				\ket{r^{ij}} = \sum_a R^{ij}_{ar} \ket{a}
			\end{equation}
			\begin{equation}
				\ket{a} = \sum_r \bar{R}^{ij}_{ar} \ket{r^{ij}}
			\end{equation}
		PNOs for a given pair are orthogonal but PNOs between pairs are non-orthogonal. One can truncate the full set of PNOs using the occupation number as a threshold; it has been found that 30 to 40 PNOs per electron pair can recover 99.9\% of canonical correlation energy for a triple-$\zeta$ basis set. Unfortunately the number of PNOs scales with the number of pairs so the total number of PNOs might still be too large. To compensate one can also truncate the set of [ij] pairs based on a pair MP2 energy threshold. The PNO methods formal scaling is $\mathcal{O}(N^5)$ though the approximations described above among others have allowed for the development of near linear scaling PNOs in CCSD\cite{Riplinger2013}

		More recently Yang et al\cite{Yang2011} has combined the ideas of pair independent PAOs and pair specific PNOs and proposed an OSV method. In this method $R^{ij}_{ar}$ is found by SVD of the diagonal MP2 pair amplitudes. 
			\begin{equation}
				[R^{i\dagger} T^{ij} R^{i}]_{rs} = t^{ii}_r \delta{rs}
			\end{equation}
			\begin{equation}
				\ket{r^i} = \sum_a \ket{a}Q^i_ar
			\end{equation}
		Like PNOs, OSVs of a single OMO are orthogonal but OSVs of different OMO's are non-orthogonal. It has been shown that typically 100 OSVs are required to recover 99.8\% of correlation energy, requiring fewer orbitals than both PNO and OSV methods. Construction of OSVs scales as  $\mathcal{O}(N^4)$