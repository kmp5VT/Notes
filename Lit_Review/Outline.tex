\renewcommand{\outlineii}{enumerate}
\begin{outline}
\0\large{Literature Review Outline }

\1 Introduction

\1 Ab Initio Many Body Quantum Mechanics
  
  \2 Hartree Fock \\
  
  \2 Electronic Correlation
    \3 Many body perturbation theory
    \3 Configuration interaction theory
    \3 Couple cluster theory\\
    
  \2 Explicitly Correlated Methods
    \3 Motivation
    \3 Explicitly correlated many body perturbation theory
    \3 Explicitly correlated couple cluster methods
    
\1 Tensor Algebra Methods to reduce Computational Complexity in Quantum Chemistry

  \2Tensor Reduction methods in a Canonical Quantum Chemistry Framework
    \3 Density fitting/resolution of the identity
    \3 Pair natural orbitals/pair atomic orbitals
    \3 Tensor decomposition methods
      \4 Tensor hypercontraction
      \4 Cholesky decomposition
      \4 Canonical product
      \4 Tucker decomposition\\
      
  \2 Tensor Ans{\"a}tze to Schr{\"o}dinger like Equations
    \3 Motivation for these kinds of methods (MPS, TNS, TTNS, DMRG etc) How this section is different from the "Canonical QC" section.
      
\1 Research Plan\\
\end{outline}

\end{document}

Resolution of the identity 
 J. L. Whitten, J. Chem. Phys. 58, 4496 (1973).
 B. I. Dunlap, J. W. D. Connolly, and J. R. Sabin, J. Chem. Phys. 71, 3396 (1979).
 O. Vahtras, J. Almlof, and M. W. Feyereisen, ? Chem. Phys. Lett. 213, 514 (1993).
 F. Weigend and M. Haser, ? Theor. Chem. Acc. 97, 331 (1997).

Cholesky Decomposition 
 S. Wilson, Comput. Phys. Commun. 58, 71 (1990).
 I. R�eggen and E. Wisl�ff-Nilssen, Chem. Phys. Lett. 132, 154 (1986).
 N. H. F. Beebe and J. Linderberg, Int. J. Quantum Chem. 12, 683 (1977).
 H. Koch, A. S. de Mers, and T. B. Pedersen, J. Chem. Phys. 118, 9481 (2003).

CP Decomposition in Quantum Chemistry 
First 2 electronic structure Theory, Second 2 Vibrational coupled cluster theory
 U. Benedikt, K.-H. Bohm, and A. A. Auer, ? J. Chem. Phys. 139, 224101 (2013).
 I. H. Godtliebsen, B. Thomsen, and O. Christiansen, J. Phys. Chem. A 117, 7267 (2013).
 I. H. Godtliebsen, M. B. Hansen, and O. Christiansen, J. Chem. Phys. 142, 024105 (2015).
 
 Tensor HyperContraction
 In first source use ALS to decompose the 3 index resolution of the identity step
 In second souce use a grid based approach and a least squares procedure
 E. G. Hohenstein, R. M. Parrish, and T. J. Mart??nez, J. Chem. Phys. 137, 044103 (2012).
 E. G. Hohenstein, R. M. Parrish, C. D. Sherrill, and T. J. Mart??nez, J. Chem. Phys. 137, 221101 (2012).
 C. Song and T. J. Mart??nez, J. Chem. Phys. 144, 174111 (2016).
 They show in the following source that for a polynomial basis on a certain grid the procedure is exact, not gauranteed using CP. Though most QC uses non polynomial basis sets
 R. M. Parrish, E. G. Hohenstein, N. F. Schunck, C. D. Sherrill, and T. J. Mart??nez, Phys. Rev. Lett. 111, 132505 (2013).